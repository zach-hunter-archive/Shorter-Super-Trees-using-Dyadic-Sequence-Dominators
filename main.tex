\documentclass{article}
\usepackage[utf8]{inputenc}
\usepackage{amsmath}
\usepackage{amssymb}
\usepackage{amsthm}
\usepackage{xcolor}
\usepackage{bm}
\usepackage{hyperref}
\usepackage{xparse}
\usepackage{mathrsfs}


\hypersetup{
    colorlinks=true,
    linkcolor=blue,
    filecolor=magenta,
    urlcolor=blue,
    citecolor=blue
}

\newtheorem{thm}{Theorem}
\newtheorem{bnd}{Bound}

\NewDocumentCommand{\snewtheorem}{m o m}
 {
  \IfNoValueTF{#2}
   {
    \newtheorem{sec#1}{#3}[section]
    \newtheorem{subsec#1}{#3}[subsection]
   }
   {
    \newtheorem{sec#1}[sec#2]{#3}
    \newtheorem{subsec#1}[subsec#2]{#3}
   }
  \newenvironment{#1}
   {\ifnum\value{subsection}>0
      \csname subsec#1\expandafter\endcsname
    \else
      \csname sec#1\expandafter\endcsname
    \fi}
   {\ifnum\value{subsection}>0 
      \csname endsubsec#1\expandafter\endcsname
    \else
      \csname endsec#1\expandafter\endcsname
    \fi}
}

\snewtheorem{lem}{Lemma}
\snewtheorem{rmk}{Remark}
\snewtheorem{cor}{Corollary}
\snewtheorem{red}{Reduction}


\DeclareMathOperator{\minto}{\bm{\mathsf{minto}}}
\DeclareMathOperator{\minout}{\bm{\mathsf{minout}}}
\DeclareMathOperator{\out}{\bm{\mathsf{out}}}
\DeclareMathOperator{\into}{\bm{\mathsf{into}}}
\DeclareMathOperator{\Conn}{\bm{\mathsf{Conn}}}

\newcommand{\hide}[1]{}
\newcommand{\edit}[1]{}%\textcolor{red}{#1}}
\newcommand{\rough}[1]{}%\textbf{\textcolor{blue}{#1}}}
\definecolor{darkgreen}{RGB}{75,150,75}
\newcommand{\review}[1]{}%\textcolor{darkgreen}{#1}}
\newcommand{\dc}[1]{}%\textcolor{orange}{dc: #1}}
\newcommand{\zh}[1]{}%\textcolor{blue}{zh: #1}}


\title{Shorter Super-Trees using Dyadic Sequence Dominators}
\author{Zach Hunter}
\date{July 2020}

\begin{document}

\maketitle
\section{Introduction}

In 2019, Defant, Kravitz and Sah introduced several ``super-tree'' problems. \hide{arXiv:1908.03197v2} In this paper, we provide a polynomial upper bound for X, when previous the upper bound was of the form $k^{\frac{1}{2}\log(k)(1+o(1))}$.

This is done by focusing on something that we will call sequence dominators, which were a key tool in the paper of Defant et al.


\section{Preliminaries}

Define d-ary trees.

Through this paper, a word $\textbf{a} = a_1,a_2,\dots a_n$ is a finite sequence of positive integers. For a word, $\textbf{a} = a_1,a_2,\dots a_n$, we say $|\textbf{a}| := n$ , $\ell(\textbf{a}) := \sum_{t=1}^n a_t$, and $M(\textbf{a}):= \max_{1 \le i \le n}\{a_i\}$.

Let $\textbf{a} =a_1,\dots,a_n,\textbf{b} = b_1,\dots,b_m$ be words. We say that $\textbf{a}\preceq \textbf{b}$ (or $\textbf{b}$ dominates $\textbf{a}$) if there exists $1 \le x_1<x_2\dots <x_i\le m$, such that $b_{x_j} \ge a_j$ for $j < i$, and $b_{x_i} \ge \sum_{t=i}^n a_t$, and we say that $x_1,\dots x_i$ is an embeds $\textbf{a}$ into $\textbf{b}$. Similarly, if there exists  a sequence $1\le x_1 < \dots <x_n\le m$ that embeds $\textbf{a}$ into $\textbf{b}$, then we say that $\textbf{a}\le \textbf{b}$.


We now will explain the motivation for this definition.

Let $A_k$ be the set of words $\textbf{a}$ such that $\ell(\textbf{a})=k$, and let $A_k'$ be the subset of $A_k$ of words $\textbf{a}$ such that $M(\textbf{a}) \le k/2$. (the words in $A_k$ are supposed to correspond to a sequence of values $s_i$ in the siphon argument of theorem 2.5. for niceness, we've ignored that some vertices belong to the trunk, instead pretending the size of the branches sum to $k$.)

Finally, we will write $W\le \textbf{b}$ if we have $\textbf{w} \le \textbf{b}$ for all $\textbf{w} \in W$. Similarly $W \preceq \textbf{b} \iff \textbf{w} \preceq \textbf{b} \quad \forall \, \textbf{w} \in W$.

Then, converting theorem 2.5 into our own language. 

\textbf{Theorem:} Let $\textbf{b}$, $|\textbf{b}|=m$, be a sequence such that $\textbf{a}\preceq \textbf{b}$ for all $\textbf{a} \in A_k'$. Then, letting $\xi_i$ denote an $i$-universal tree for all $i<k$, we can construct a $k$-universal tree like so. Start with the $m-1$-spine. Glue $\xi_{b_i}$ to the left and right leaves of the $i$-th vertebra, for $i < m$. Then, glue $\xi_{b_m}$ to the center leaf of the $m-1$-th vertebra.


Brief proof? (perhaps sketch ideas and then put proper proof in appendix?)


Let $B_n$ denote the set of words $\textbf{b}$ such that $M(\textbf{b})<n$ and $\textbf{a}\preceq \textbf{b}$ for all $\textbf{a} \in A'_n$.


\textbf{Definition:} for a function $f: \textbf{N}\to \textbf{R}$ and word $\textbf{b}$, we say $f(\textbf{b}):= \sum_{t=1}^{|\textbf{b}|} f(b_t)$.

\textbf{Question 1:} what is the asymptotically largest function $f_-$ with $f_-(1) =1$, such that for all $n$, if $\textbf{b}\in B_n$, we have $f_-(\textbf{b}) \ge  f_-(n)$.

\textbf{Question 2:} what is the asymptotically smallest function $f_+$ with $f_+(1) = 1$, such that for all $n$, there exists a word $\textbf{b}\in B_n$ where $f_+(\textbf{b}) \le  f_+(n)$.

In the following sections we will show that $f_-(n) = \Omega\left(\frac{n^2}{\log_2(n)}\right)$, and $f_+(n) = O(n^{\log_2(7)})$.


\section{Upper Bound}

For a word $\textbf{a} = a_1,\dots ,a_n$, we say $\textbf{a}[i,j] = a_i,\dots, a_j$. For two words $\textbf{a},\textbf{b}$ of lengths $n,m$ respectively, we denote $(\textbf{a}+\textbf{b})$ as a word with $n+m$ letters such that $(\textbf{a}+\textbf{b})[1,n] = \textbf{a}, (\textbf{a}+\textbf{b})[n+1,n+m] = \textbf{b}$.

Let $f(\textbf{a},\textbf{b}) = \sum_{t=1}^i a_t$ where $i$ is the maximum integer such that $\textbf{a}[1,i] \le \textbf{b}$.

\textbf{Remark 1:} $f(\textbf{a}_1+\textbf{a}_2,\textbf{b}) \ge f(\textbf{a}_1,\textbf{b})\le f(\textbf{a}_1,\textbf{a}_1) = \ell(\textbf{a}_1)$ and equality must hold on at least one side.

\textbf{Proof:} The inequalities are trivial. Meanwhile, we can prove the claim about equality like so. Suppose that equality does not hold in the left inequality. It follows that $\textbf{b}$ cannot dominate $\textbf{a}_1$. In turn, this implies that $b$ cannot dominate $\textbf{a}_1 +\textbf{a}_2$ any better. 

\textbf{Remark 2:} suppose $x\ge y\ge z$, it follows that $f(\textbf{a}+[y],\textbf{b}+[x]) \ge f(\textbf{a}+[z],\textbf{b}+[x])$.

\textbf{Proof:} Case 1: $\textbf{a} \not \le \textbf{b}$, we can at most embed $i<n$ letters of $\textbf{a}$ into $\textbf{b}$. It follows that then $f(\textbf{a}+[y],\textbf{b}+[x])  =f(\textbf{a}+[z],\textbf{b}+[x])  = f(\textbf{a},\textbf{b}+[x])$. 

Case 2: If $\textbf{a}\le \textbf{b}$, then since $x \ge y\ge z$, we can the last letter in $\textbf{a}+[y]$ into the last letter of $\textbf{b}+[x]$, giving $\textbf{a}+[z]\le \textbf{a}+[y]\le \textbf{b}+[x]$. Thus $f(\textbf{a}+[y],\textbf{b}+[x]) = \sum_{t=1}^n a_t +y = f(\textbf{a}+[z],\textbf{b}+[x]) +(y-z)$. As $y \ge z$ we are done. $\blacksquare$


We let $\textbf{bin}(n) = b_1,\dots b_{2^n}$ where $b_i$ is the largest power of $2$ which divides $i$.



\textbf{Remark 3:} $\textbf{bin}(n)\le \textbf{bin}(n+1)[2^n+1,2^{n+1}]$.



\textbf{Remark 4:} for $n \ge 0$, we have $f(\textbf{a},\textbf{bin}(n))\ge 2^{n-1} $ for all $\textbf{a} \in A_{2^{n}}$

\textbf{Proof:} We proceed by induction. This is trivially  when $n = 0$ as $A_1 = \{[1]\},\textbf{bin}(0) = [1]$.

We now consider $n \ge 1$. Suppose for sake of contradiction there was $\textbf{a} \in A_{2^n}$ such that $f(\textbf{a},\textbf{bin}(n))<2^{n-1}$. There must exist a minimal $i$ such that $\sum_{t=1}^{i} a_t \ge 2^{n-1}$. 

By Remark 1, if $\textbf{a}$ is a counter-example to our claim, we must have $f(\textbf{a}[1,i],\textbf{b})<2^{n-1}$. WLOG, we may assume $\sum_{t=1}^i a_t = 2^{n-1}$, as $a_i \le 2^n = \textbf{bin}(n)[n]$, allowing us to apply Remark 2.

Now, there must be a minimal $i_2$ such that $\sum_{t=1}^{i_2} a_t \ge 2^{n-2}$. Suppose for sake of contradiction that $\textbf{a}[1,i_2] \not\le \textbf{bin}(n)[1,2^{n-1}] = \textbf{bin}(n-1)$. By minimality of $i_2$ it follows that $f(\textbf{a}[1,i_2],\textbf{bin}(n-1)) < 2^{n-2}\le \ell(\textbf{a}[1,i_2])$. Thus, by Remark 1, we must have that $f(\textbf{a}[1,i],\textbf{b}) = f(\textbf{a}[1,i_2],\textbf{b})<2^{n-2}$. This contradicts the inductive hypothesis as $\textbf{a}[1,i] \in A_{2^{n-1}}$. Thus, we must have $\textbf{a}[1,i_2]\le \textbf{bin}(n)[1,2^{n-1}]$.

Repeating the above argument, we must have that $\textbf{a}[i_2+1,i] \le \textbf{bin}(n)[2^{n-1}+1,2^n]$. (making use of Remark 3)

\textbf{Theorem 1:} $\textbf{a} \preceq \textbf{bin}(n-1)+\textbf{bin}(n-1)+2^{n-1}$ for all $\textbf{a} \in A'_{2^n}$.

Suppose for sake of contradiction, that this did not hold for some $\textbf{a} \in A'_{2^n}$, for some $n$. Obviously, by definition of $A'$, we know $\ell(\textbf{a}) = 2^n$ and $M(\textbf{a}) \le 2^{n-1}$. 

Let $|\textbf{a}| = m$. It is clear that there exists $1\le i<j \le m$ such that $\ell(\textbf{a}[1,i-1])<2^{n-2}\le \ell(\textbf{a}_1:=\textbf{a}[1,i])$ and $\ell(\textbf{a}[i+1,j-1])<2^{n-2}\le \ell(\textbf{a}_2:=\textbf{a}[i+1,i+j])$. By Remark 4, it follows that $\textbf{a}[1,i] \le \textbf{bin}(n-1),\textbf{bin}[i+1,i+j] \le \textbf{bin}(n-1)$. Furthermore, we must have that $\ell(\textbf{a}_3:=\textbf{a}[i+j+1,m]) \le 2^n-2^{n-2}-2^{n-2} = 2^{n-1}$, recalling the lengths of $\textbf{a},\textbf{a}_1$ and $\textbf{a}_3$. 

Hence, we may embed $\textbf{a}_1$ into the first copy of $\textbf{bin}(n-1)$, by Remark 4. For the same reasons, $\textbf{a}_2$ can be embedded in the second copy of $\textbf{bin}(n-1)$. And then, finally we can tail-embed $\textbf{a}_3$ into $2^{n-1}$. Hence $\textbf{a} \preceq \textbf{bin}(n-1)+\textbf{bin}(n-1)+2^{n-1}$ as desired.


\textbf{Lower Bound}

We will now set a lower bound on the extent to which this strategy can go. 

For a word $\textbf{a} = a_1, \dots, a_t$, let $Sort(\textbf{a}) := a_{\sigma(1)}, \dots a_{\sigma(t)}$ where $\sigma$ is a (not necessarily unique) permutation such that $a_{\sigma(1)}\le a_{\sigma(2)} \le  \dots a_{\sigma(t)}$. Furthermore for a set of words $X$, let $Sort(X):= \{ Sort(\textbf{w}): \textbf{w} \in X\}$.

Obviously, as $Sort(A_n') \subseteq A_n'$, if $\textbf{b} \in B_n$, then $\textbf{a} \preceq \textbf{b}$ for all $\textbf{a} \in Sort(A_n')$. Thus, defining $B_n^\bigstar $ as the words $\textbf{b} \in [n-1]^*$ such that $\textbf{a} \preceq \textbf{b}$ for all $\textbf{a} \in Sort(A_n')$, we have $B_n^\bigstar \supseteq B_n$.

Hence, with $f(1) = 1, f(n):= \min\{f(\textbf{b}): b \in B_n\}$, and $g(1) = 1, g(n) := \min\{g(\textbf{b}): \textbf{b} \in B_n^\bigstar\}$, we have $f(n) \geq g(n)$ for all $n$.



\textbf{Remark:} if $Sort(\textbf{a}) \preceq \textbf{b}$ then $Sort(\textbf{a}) \preceq Sort(\textbf{b})$, and same similarly holds for $\le$


Now, let $\textbf{log}(n):= Sort( \lfloor n/1 \rfloor,\lfloor n/2 \rfloor, \dots  \lfloor n/(n-1) \rfloor, \lfloor n/n \rfloor)$. We shall prove that $Sort(A_n) \le \textbf{log}(n)$ and $Sort(A_n) \le \textbf{b} \implies \textbf{log}(n)  \le \textbf{b}$. (we have that this function has growth rate of roughly $n^{9/5}$)

Cases: largest element is large - idea: g(n/2) + g(n-k)

largest element isn't large - idea g(\textbf{logg}(k))+g(n-k)

There exists $C$ such that $g(n) \ge C n^{1.4}$ for all $n \in [1,3500]$. For $n\ge 900, i \in [1,300], \lfloor n/i\rfloor \ge \lceil n/2i\rceil $. Thus, $g(\lfloor n/i \rfloor) \ge  g(\lceil n/2i\rceil ) \ge C(n/2i)^{1.4} \ge C/4 (n/i)^{1.4}$, as $g$ is increasing.

By strong induction, it follows that $g(n) \ge C/4 n^{1.4}$ for all $n \ge 1$.

\textbf{Proof:} Suppose $Sort(A_n')\preceq \textbf{b}$. Let $\textbf{b} = \textbf{b}'+n-i$.

If $i \le 0.26 n$, then as $\lfloor n/2\rfloor \le \textbf{b}'$, thus\[\implies g(\textbf{b}) \ge g(\lfloor n/2 \rfloor) +g(n-i)\ge C/4( (n/2)^{1.47}+ (n-i)^{1.47}) \]\[\ge C/4 n^{1.4}( (1/2)^{1.4}+ (0.72)^{1.4}) =  C/4 n^{1.47}(1.00333\dots) \ge C/4 n^{1.4}.\]

Meanwhile, if $i \ge 0.26 n > 900 $, then $A_i' \le \textbf{b}' \le \textbf{logg}(i)$. Now,

\[g(\textbf{log}(i)+n-i) \ge C/4 n^{1.47} \left((i/n)^{1.47}\sum_{k=1}^{300} \frac{1}{k^{1.47}} + (1-i/n)^{1.47}\right).\]

Treating $i/n$ as variable $r \in [0.26,0.5]$, we can easily verify that $r^{1.47} \sum_{k=1}^{300} \frac{1}{k^{1.47}} + (1-r)^{1.47} \ge 1$ on this domain, finishing our result. A quick check on WolframAlpha reveals that this inequality is positive for $r\in[0.25973,1]$, which is more than sufficient. $\blacksquare$

We note that letting $h(1) =1, h(n) = \min_{1\le i \le n/2} \{ h(n-i)+\min\{h(\textbf{log}(i)),h(\lfloor n/2\rfloor )\}\}$, a quick numeric check seems to to indicate that $h(n) = \Theta(n^{\log(30)/\log(10)}) \approx n^{1.477\dots}$. Thus, this seems close to the best possible one can achieve without more sophisticated ideas.


Finally, we note that defining $\textbf{logg}(n)$ to be $\textbf{log}(n)$ except with the largest letter being replaced by $\lfloor n/2\rfloor$, we have that $A_n' \le \textbf{logg}(n)$ and $A_n' \le \textbf{b} \implies \textbf{logg}(n)\le \textbf{b}$.

We have that this gives an upper bound for $g$, of $O(n^x)$, where $x$ satisfies $1/2^x+ \sum_{k=2}^\infty 1/k^x \le 1\iff \zeta(x) +1/2^x\le 2 $ thus we may take $x = 1.9133$.


For $\textbf{b}'$, the $t$-th largest letter $b_t$, we should have that $(t-1)(b_t+1) \ge i$ or $(t+1)(b_t+1) > n$ and $t(b_t) \ge i$. Thus $b_t+1 \ge \min\{ rn/(t-1), n/(t+1)\}$ and $b_t \ge rn/t$

Now, $Sort(A_n')\preceq Sort(n/2,n/3 + \textbf{log}(n/2))) $. This gives a yeild to a bound of $g(n) = O(n^{1.77})$, as $x=1.77$ satisfies $1/2^x + 1/3^x +(1/2^x)\zeta(x)<1$.

\end{document}